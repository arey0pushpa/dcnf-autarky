\documentclass{article}
\usepackage[utf8]{inputenc}
\usepackage[english]{babel}
 
\usepackage{natbib}
\usepackage{lmodern}
\usepackage{amsmath}

%\input{macro}

\begin{document}

\title{DQBF Survey}

\author{
    Ankit Shukla
}

%\institute{JKU}

% \author{to be announced}

\maketitle

\begin{abstract}
We survey the solving approaches for DQBF, preprocessing and inprocessing techniques,
and further advances in DQBF solving.

%%% Local Variables:
%%% mode: latex
%%% TeX-master: "main"
%%% End:

\end{abstract}

\section{Introduction}
\label{sec:intro}
\begin{enumerate}
  \item Hentikka's quantifier: Although very important. Not relevant for our discussion.
  \item Henkin's quantifier:
  According to the wikipedia, ``In logic a branching quantifier, also called a Henkin quantifier, finite partially ordered quantifier $\langle Qx_{1}\dots Qx_{n}\rangle$ 
  of quantifiers for $Q  \in \{\forall,\exists\}$", so we should also points the name ``Henkin's" or ``Branching".
   
  The simplest Henkin quantifier, $Q_{H}$ is :
  \begin{displaymath}
  \binom{\forall x_{1} \exists y_{1}}{\forall x_{2} \exists y_{2}} \, \phi(x_{1},x_{2},y_{1},y_{2})
  \end{displaymath}
  
  ``Branching quantification first appeared in a 1959 conference paper of Leon Henkin~\cite{henkin1961some}.
  Systems of partially ordered quantification are intermediate in strength between first-order logic and second-order logic".
  
  In computer science literature first used in~\cite{peterson1979multiple} to model multiple
  person (team) games of incomplete information.
  The generalization of the alternation machines (nondeterministic
  Turing Machines with existential and universal quantifiers alternation, conceptually similar to QBF) to classes of machines namely ``\textit{multiple person
  alternation machines}" (conceptually DQBF).
  
\end{enumerate}

\section{First solving approach: DQDPLL}
The first solving approach was presented in~\cite{frohlich2012dpll}.
%
This was adaptation of QDPLL from QBF to DQBF, e.g., unit propagation, clause
learning, universal reduction, watched literals, etc.
%

The motivation was to solve the quantifier-free bit-vector formulas (QF\_BV) by translating it to DQBF (both belongs to same complexity class).

\section{First Application: PEC Problems}
The application of partial equivalence checking (PEC) of circuits transformed as a DQBF satisfiability was presented in~\cite{gitina2013equivalence}.
%
The presented algorithm solve DQBF based on variable elimination (\cite{biere2004resolve}).

\section{First publicly available solver}
The technique presented in~\cite{finkbeiner2014fast} is similar to BMC encoding of QBF and can only solve UNSAT problems.  

\section{Complete solver: iDQ}
The iDQ solver presented in~\cite{frohlich2014idq} adapts and extends the Inst-Gen approach (solving approach for EPR logic, fragment of first order logic) to DQBF.

\section{A new solver: HQS}
The work of~\cite{gitina2015solving} present an improved expansion-based solver, HQS.
%
The solver expands DQBF to QBF, eliminates the minimum set of variables that cause non-linear
dependencies and uses AIGs to detect units and pure literals.

\section{Preprocessing and inprocessing}
The work of~\cite{wimmer2015preprocessing} generalize preprocessing techniques for QBF to DQBF. 
%
The technique includes, blocked clause elimination, equivalence reasoning, structure extraction, and variable elimination by resolution.

In further work~\cite{wimmer2017hqspre} presents a preprocessor HQSpre for both QBF's and DQBF's.
%
The techniques, variable elimination routines, clause elimination routines, clause strengthening routines, dependency schemes and gate rewriting.

\section{Dependency}

\section{Proof systems}

\section{Certification perspective and Skolem functions}

\begin{enumerate}
	\item Henkin Quantifiers and Boolean Formulae~\cite{balabanov2012henkin}
	
	\item A certification perspective of DQBF~\cite{balabanov2014henkin}:
	
	Two central references in both are~\cite{bubeck2006dependency} and ~\cite{bubeck2010model}. 
\end{enumerate}

The work by~\cite{wimmer2016skolem} obtains Skolem functions from an elimination-based DQBF solver HQS, while taking preprocessing steps into account.
%
The size of the Skolem functions is optimized by don’t-care minimization using Craig interpolants and rewriting techniques.

%\bibliographystyle{spbasic}      % basic style, author-year citations
%\bibliographystyle{spmpsci}      % mathematics and physical sciences
%\bibliographystyle{spphys}       % APS-like style for physics
\bibliographystyle{unsrtnat}
\bibliography{biblio}

\end{document}

%--------------------- DO NOT ERASE BELOW THIS LINE --------------------------

%%% Local Variables:
%%% mode: latex
%%% TeX-master: "main"
%%% End:
