\documentclass{article}
\usepackage[utf8]{inputenc}
\usepackage[english]{babel}
 
\usepackage{natbib}
\usepackage{lmodern}

%\input{macro}

\begin{document}

\title{DQBF Survey}

\author{
    Ankit Shukla
}

%\institute{JKU}

% \author{to be announced}

\maketitle

\begin{abstract}
We survey the solving approaches for DQBF, preprocessing and inprocessing techniques,
and further advances in DQBF solving.

%%% Local Variables:
%%% mode: latex
%%% TeX-master: "main"
%%% End:

\end{abstract}

\section{Introduction}
\label{sec:intro}
\begin{enumerate}
  \item Henkin's quantifier
  \item Hentikka's quantifier
\end{enumerate}

\section{First solving approach: DQDPLL}
The first solving approach was presented in~\cite{frohlich2012dpll}.
%
This was adaptation of QDPLL from QBF to DQBF, e.g., unit propagation, clause
learning, universal reduction, watched literals, etc.
%

The motivation was to solve the quantifier-free bit-vector formulas (QF\_BV) by translating it to DQBF (both belongs to same complexity class).

\section{First Application: PEC Problems}
The application of partial equivalence checking (PEC) of circuits transformed as a DQBF satisfiability was presented in~\cite{gitina2013equivalence}.
%
The presented algorithm solve DQBF based on variable elimination (\cite{biere2004resolve}).

\section{First publicly available solver}
The technique presented in~\cite{finkbeiner2014fast} is similar to BMC encoding of QBF and can only solve UNSAT problems.  

\section{Complete solver: iDQ}
The iDQ solver adapts and extends the Inst-Gen approach (solving approach for EPR logic, fragment of first order logic) to DQBF and was presented in~\cite{frohlich2014idq}.

\section{A new solver: HQS}
The work of~\cite{gitina2015solving} present an improved expansion-based solver, HQS.
%
The solver expands DQBF to QBF, eliminates the minimum set of variables that cause non-linear
dependencies and uses AIGs to detect units and pure literals.

\section{Preprocessing and inprocessing}
The work of~\cite{wimmer2015preprocessing} generalize preprocessing techniques for QBF to DQBF. 
%
The technique includes, blocked clause elimination, equivalence reasoning, structure extraction, and variable elimination by resolution.

The work by~\cite{wimmer2017hqspre} presents a preprocessor HQSpre for both QBF's and DQBF's.
\begin{enumerate}
 \item Variable Elimination Routines
 \item Clause Elimination Routines.
 \item Clause Strengthening Routines. 
 \item Dependency schemes
 \item Gate rewriting
\end{enumerate}
%\bibliographystyle{spbasic}      % basic style, author-year citations
%\bibliographystyle{spmpsci}      % mathematics and physical sciences
%\bibliographystyle{spphys}       % APS-like style for physics
\bibliographystyle{unsrtnat}
\bibliography{biblio}

\end{document}

%--------------------- DO NOT ERASE BELOW THIS LINE --------------------------

%%% Local Variables:
%%% mode: latex
%%% TeX-master: "main"
%%% End:
