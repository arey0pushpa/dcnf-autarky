\documentclass{article}
\usepackage[utf8]{inputenc}
\usepackage[english]{babel}
 
%\usepackage{natbib}
\usepackage{todonotes}
\usepackage[driverfallback=hypertex]{hyperref}
\usepackage{lmodern}
%\usepackage{amsmath}
\usepackage{amsmath, amsthm, amssymb}
\usepackage[normalem]{ulem}

\newtheorem{prop}{Proposition}

%\usepackage{mathtools}
\usepackage{makecell}
\usepackage{graphicx}
\usepackage{amsmath}
\usepackage{pdflscape}
\usepackage{rotating}
% \usepackage[top=0.85in,left=2.75in,footskip=0.75in]{geometry}

% I ADDED FOR THE CHANGE IN ENUMERATE TO ALPHABBETICAL
\usepackage{enumitem}


% amsmath and amssymb packages, useful for mathematical formulas and symbols
\usepackage{amsmath,amssymb}

\renewcommand{\figurename}{Fig.{}}

% Use adjustwidth environment to exceed column width (see example table in text)
\usepackage{changepage}

% Use Unicode characters when possible
\usepackage[utf8x]{inputenc}

% textcomp package and marvosym package for additional characters
\usepackage{textcomp,marvosym}

% cite package, to clean up citations in the main text. Do not remove.
\usepackage{cite}

% Use nameref to cite supporting information files (see Supporting Information section for more info)
\usepackage{nameref,hyperref}

% line numbers
\usepackage[right]{lineno}

% ligatures disabled
\usepackage{microtype}
\DisableLigatures[f]{encoding = *, family = * }

% color can be used to apply background shading to table cells only
\usepackage[table]{xcolor}

\usepackage{todonotes}

% array package and thick rules for tables
\usepackage{array}

% Use package Listing to add code in our Manuscript
\usepackage{listings} 

% Added for the sub-pictures
\usepackage{subcaption}

% Added for the multi-column 
\usepackage[british]{babel}
\usepackage{hhline}
\usepackage{multirow}
\usepackage[figurename=Fig]{caption}

%ADEDED BY ANKIT
\usepackage{tkz-orm}
\usepackage{lineno}
\linenumbers
%%%<
%\usepackage[active,tightpage]{preview}
%\PreviewEnvironment{tikzpicture}

\usepackage{verbatim}
\usepackage{pgfplots}
\newcommand*{\equal}{=}
 \usepackage{tikz}
 \usetikzlibrary{arrows}
 \usepackage{xparse}
\usetikzlibrary{matrix,backgrounds}
\pgfdeclarelayer{myback}
\pgfsetlayers{myback,background,main}

\tikzset{mycolor/.style = {line width=1bp,color=#1}}%
\tikzset{myfillcolor/.style = {draw,fill=#1}}%


\NewDocumentCommand{\highlight}{O{blue!40} m m}{%
\draw[mycolor=#1] (#2.north west)rectangle (#3.south east);
}

\NewDocumentCommand{\fhighlight}{O{blue!40} m m}{%
\draw[myfillcolor=#1] (#2.north west)rectangle (#3.south east);
}
 \usetikzlibrary{matrix,decorations.pathreplacing, calc, positioning}



% create "+" rule type for thick vertical lines
\newcolumntype{+}{!{\vrule width 2pt}}
\renewcommand{\thesubfigure}{\Alph{subfigure}}

% create \thickcline for thick horizontal lines of variable length
\newlength\savedwidth
\newcommand\thickcline[1]{%
  \noalign{\global\savedwidth\arrayrulewidth\global\arrayrulewidth 2pt}%
  \cline{#1}%
  \noalign{\vskip\arrayrulewidth}%
  \noalign{\global\arrayrulewidth\savedwidth}%
}

\usepackage{array}
\newcolumntype{L}[1]{>{\raggedright\let\newline\\\arraybackslash\hspace{0pt}}m{#1}}
\newcolumntype{C}[1]{>{\centering\let\newline\\\arraybackslash\hspace{0pt}}m{#1}}
\newcolumntype{R}[1]{>{\raggedleft\let\newline\\\arraybackslash\hspace{0pt}}m{#1}}


% Remove comment for double spacing
%\usepackage{setspace} 
%\doublespacing

% Text layout
% \raggedright
% \setlength{\parindent}{0.5cm}
% \textwidth 5.25in 
% \textheight 8.75in
% create "+" rule type for thick vertical lines
% \newcolumntype{+}{!{\vrule width 2pt}}
\renewcommand{\thesubfigure}{\Alph{subfigure}}

% \thickhline command for thick horizontal lines that span the table
\newcommand\thickhline{\noalign{\global\savedwidth\arrayrulewidth\global\arrayrulewidth 2pt}%
\hline
\noalign{\global\arrayrulewidth\savedwidth}}

% \raggedright
% \setlength{\parindent}{0.5cm}
% \textwidth 5.25in 
% \textheight 8.75in


% \usepackage[aboveskip=1pt,labelfont=bf,labelsep=period,justification=raggedright,singlelinecheck=off]{caption}
% \renewcommand{\figurename}{Fig}
% \usepackage{epstopdf}
% \AppendGraphicsExtensions{.tif}

\newcommand{\booleans}{\mathbb{B}}
\newcommand{\naturals}{\mathbb{N}}
\newcommand{\integers}{\mathbb{Z}}
\newcommand{\ordinals}{\mathbb{O}}
\newcommand{\numarals}{\mathbb{I}}

\newcommand{\maps}{\rightarrow}

\newcommand{\union}{{\cup} }
\newcommand{\Union}{{\bigcup} }
\newcommand{\powerset}[1]{2^{#1}}
\newcommand{\intersection}{{\cap} }
\newcommand{\intersect}{\intersection}
\newcommand{\Intersection}{{\bigcap} }
\newcommand{\compose}{{\circ} }


\newcommand{\ltrue}{\mathbf{tt}}
\newcommand{\lfalse}{\mathbf{ff}}
\newcommand{\limplies}{\Rightarrow}
\newcommand{\lxor}{\oplus}
\newcommand{\Land}{\bigwedge}
\newcommand{\Lor}{\bigvee}
\newcommand{\Lxor}{\bigoplus}
\newcommand{\lequiv}{\Leftrightarrow}
\newcommand{\landplus}{\mathrel{:\hspace{-3pt}\land\hspace{-3pt}=}}
\newcommand{\lorplus}{\mathrel{:\hspace{-3pt}\lor\hspace{-3pt}=}}

\newcommand{\nodes}{N}
\newcommand{\mols}{M}
\newcommand{\nlabel}{L}
\newcommand{\edges}{E}
\newcommand{\pairs}{\mathcal{P}}
\newcommand{\nodef}{f}
\newcommand{\edgef}{g}


\newcommand{\lorem}{{\bf LOREM}}
\newcommand{\ipsum}{{\bf IPSUM}}

\newcommand{\zthree}{\textsc{Z3}}
\newcommand{\ourtool}{\textsc{VTSSynth}}
\newcommand{\depqbf}{\textsc{DepQBF}}
\newcommand{\ashu}[1]{ {\textcolor{magenta} {Ashutosh: #1}} }
\newcommand{\mukund}[1]{ {\textcolor{red} {Mukund: #1}} }
\newcommand{\srivas}[1]{ {\textcolor{blue} {Srivas: #1}} }
\newcommand{\ankit}[1]{ {\textcolor{green!50!black}{Ankit: #1}} }

\newtheorem{df}{Definition}

%--------------------- DO NOT ERASE BELOW THIS LINE --------------------------

%%% Local Variables:
%%% mode: latex
%%% TeX-master: "main"
%%% End:


\begin{document}

\title{DQBF Survey}

\author{
    Ankit Shukla
}

%\institute{JKU}

% \author{to be announced}

\maketitle

\begin{abstract}
We survey the solving approaches for DQBF, preprocessing and inprocessing techniques,
and further advances in DQBF solving.

%%% Local Variables:
%%% mode: latex
%%% TeX-master: "main"
%%% End:

\end{abstract}

\section{Introduction}
\label{sec:intro}
\begin{enumerate}
  \item Hentikka's quantifier: Although very important. Not relevant for our discussion.
  \item Henkin's quantifier:
  According to the wikipedia, ``In logic a branching quantifier, also called a Henkin quantifier, finite partially ordered quantifier $\langle Qx_{1}\dots Qx_{n}\rangle$ 
  of quantifiers for $Q  \in \{\forall,\exists\}$", so we should also points the name ``Henkin's" or ``Branching".
   
  The simplest Henkin quantifier, $Q_{H}$ is :
  \begin{displaymath}
  \binom{\forall x_{1} \exists y_{1}}{\forall x_{2} \exists y_{2}} \, \phi(x_{1},x_{2},y_{1},y_{2})
  \end{displaymath}
  
  ``Branching quantification first appeared in a 1959 conference paper of Leon Henkin~\cite{henkin1961some}.
  Systems of partially ordered quantification are intermediate in strength between first-order logic and second-order logic".
  
  In computer science literature first used in~\cite{peterson1979multiple} to model multiple
  person (team) games of incomplete information.
  The generalization of the alternation machines (nondeterministic
  Turing Machines with existential and universal quantifiers alternation, conceptually similar to QBF) to classes of machines namely ``\textit{multiple person
  alternation machines}" (conceptually DQBF).
  
\end{enumerate}

\section{First solving approach: DQDPLL}
The first solving approach was presented in~\cite{frohlich2012dpll}.
%
This was adaptation of QDPLL from QBF to DQBF, e.g., unit propagation, clause
learning, universal reduction, watched literals, etc.
%

The motivation was to solve the quantifier-free bit-vector formulas (QF\_BV) by translating it to DQBF (both belongs to same complexity class).

\section{First Application: PEC Problems}
The application of partial equivalence checking (PEC) of circuits transformed as a DQBF satisfiability was presented in~\cite{gitina2013equivalence}.
%
The presented algorithm solve DQBF based on variable elimination (\cite{biere2004resolve}).

\section{First publicly available solver}
The technique presented in~\cite{finkbeiner2014fast} is similar to BMC encoding of QBF and can only solve UNSAT problems.  

\section{Complete solver: iDQ}
The iDQ solver presented in~\cite{frohlich2014idq} adapts and extends the Inst-Gen approach (solving approach for EPR logic, fragment of first order logic) to DQBF.

\section{A new solver: HQS}
The work of~\cite{gitina2015solving} present an improved expansion-based solver, HQS.
%
The solver expands DQBF to QBF, eliminates the minimum set of variables that cause non-linear
dependencies and uses AIGs to detect units and pure literals.

\section{Preprocessing and inprocessing}
The work of~\cite{wimmer2015preprocessing} generalize preprocessing techniques for QBF to DQBF. 
%
The technique includes, blocked clause elimination, equivalence reasoning, structure extraction, and variable elimination by resolution.

In further work~\cite{wimmer2017hqspre} presents a preprocessor HQSpre for both QBF's and DQBF's.
%
The techniques, variable elimination routines, clause elimination routines, clause strengthening routines, dependency schemes and gate rewriting.

\section{Dependency}
\todo{Only read overview of the papers in this section: read completely}The work by~\cite{wimmer2016dependency} analyse the dependency
schemes proposed for QBF to apply on DQBF. 
%
The paper gives correctness proof of the few that are sound for DQBF. 
%

The \cite{wimmer2017dqbf} eliminate of dependencies of DQBF to convert it to an equisatisfiable QBF.
%
The technique is implemented in HQS.

% which result from the application of dependency schemes to the formula and can be added to or removed from the formula at no cost


The work ~\cite{beyersdorff2018reinterpreting} proposed an approach which improve the relationship between DQBF and QBF
dependency schemes.
%
Core of the paper is the method that gives rise to soundness results across different calculi.

\section{Proof systems}
The work~\cite{beyersdorff2016lifting} examines the lifting existing resolution systems of QBF to DQBF.
%
QBF have a strict chain of proof systems, where Q-Res $<$ IR-calc $<$ IRM-calc.
%

In the case of DQBF the obvious adaptations of Q-Res and universal resolution turns out to be too weak: they are not complete. 
%
The obvious adaptation of IR-calc has
the right strength: it is sound and complete. 
%
The IRM-calc and long-distance resolution is too strong: it is not sound any more.

\todo{Incomplete and only intro}The work of~\cite{rabe2017resolution} presents a sound and complete proof system for DQBF.
%
The system has three rules, resolution, universal reduction, and a new proof rule called \textit{fork extensions}.

The definition of resolution admits universal variables as pivots...

\section{Certification perspective and Skolem functions}

%\begin{enumerate}
The work of Henkin Quantifiers and Boolean Formulae~\cite{balabanov2012henkin} investigates DQBF's properties like negation, complement, formula expansion, and prenex, non-prenex form conversion. 
%
The paper used two forms to describe prenex DQBF, Skolem (S-Form) and Herbrand (H-Form) form:

\begin{align}
	\Phi_{S} \,\, = \,\, \forall x_{1}... \forall x_{n} \exists y_{1_{(S_{1})}}...\exists y_{m_{(S_{m})}} . \phi \label{eq1}\\
	\Phi_{H}  \,\, = \,\, \forall x_{1_{(H_{1})}}... \forall x_{n_{(H_{n})}} \exists y_{1}...\exists y_{m} . \phi \label{eq2}
\end{align}

Additional subscripts on the quantified variables is just another way to express dependencies, $S_{i} \subseteq A$ and $H_{i} \subseteq E$.

In the case of S-Form, $\exists y_{1_{(S_{1})}}$ means  existential variable $y_{1}$ depend on the universal variables set $S_{1}$. 
%
Similarly, for the H-form, universal quantifiers in the prefix are annotated with the existential variables they depend on. 
     
%Semantically, the truth and falsity of a DQBF is interpreted as existence of Skolem (in the case of \ref{eq1}) and Herbrand fucntion (in the case of \ref{eq2}). 
Skolem functions serve as the model to a true S-form DQBF whereas Herbrand functions serve as the countermodel to a false H-form DQBF.

Most important property discussed is the notion of \textbf{negation} and \textbf{complement}.
%
In the case of negation ($\neg$) the quantifiers are flipped and propositional formula is negated but no change in the dependency set and it's position.
 
\begin{align}
\neg \Phi_{S} \,\, =   \,\, \exists x_{1}... \exists x_{n} \forall y_{1_{(S_{1})}}...\forall y_{m_{(S_{m})}} . \neg \phi \label{eq3}
\end{align}
 
In the case of complement ($\sim$) the position is shifted from one quantifier type to another in the prefix. Here, $H^{\prime}_{i} = \{ y_{i} \in E \, | \, x_{i} \notin S_{j} \}$
 
\begin{align}
 \sim \Phi_{S} \,\, =   \,\, \exists x_{1_{(H^{\prime}_1)}}... \exists x_{1_{(H^{\prime}_n)}} \forall y_{1}...\forall y_{m} . \neg \phi \label{eq4}
\end{align}
 
Interestingly the following proposition holds, 
\begin{prop}
	DQBF under the negation operation obeys the law of excluded middle but under the complement operation it does not.
\end{prop} 

In the journal paper ``A certification perspective of DQBF"~\cite{balabanov2014henkin} additionally a generalized resolution rule for DQBF evaluation is presented. Although,  it is shown to be sound, but incomplete.
	
Two central references in both are~\cite{bubeck2006dependency} and ~\cite{bubeck2010model} in introduction section referring that S-form DQBF can be converted to an equivalent QBF by formula expansion on the universal variables.
	
%\end{enumerate}

The work by~\cite{wimmer2016skolem} obtains Skolem functions from an elimination-based DQBF solver HQS, while taking preprocessing steps into account.
%
The size of the Skolem functions is optimized by don’t-care minimization using Craig interpolants and rewriting techniques.

%\bibliographystyle{spbasic}      % basic style, author-year citations
%\bibliographystyle{spmpsci}      % mathematics and physical sciences
%\bibliographystyle{spphys}       % APS-like style for physics
%\bibliographystyle{unsrt}
\bibliographystyle{apalike}
\bibliography{biblio}

\end{document}

%--------------------- DO NOT ERASE BELOW THIS LINE --------------------------

%%% Local Variables:
%%% mode: latex
%%% TeX-master: "main"
%%% End:
