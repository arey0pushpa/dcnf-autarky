% Oliver Kullmann, 22.2.2020 (Swansea)
% The bibtex-entry of this paper is \cite{KullmannShukla2020}.
% Submitted to SAT 2020:
% XXX

% Switching between report- and journal-version: uncomment the appropriate
% block of definitions of \Schrift, \Liste.

\documentclass[runningheads]{llncs}

%\input Latex_macros/Definitionen.tex

\usepackage{enumerate}
\usepackage[active]{srcltx}
\usepackage[all]{xy}

\usepackage{xr}

% Added By Ankit
\usepackage{multirow}
\usepackage{hyperref}



\begin{document}

\title{Autarky Tables}

\author{Oliver Kullmann\inst{1} \and Ankit Shukla\inst{2}}
\authorrunning{O.~Kullmann, A.~Shukla}

\institute{Swansea University, Swansea, UK\\
  \email{O.Kullmann@Swansea.ac.uk}
\and
    Johannes Kepler University, Austria\\
    \email{ankit.shukla@jku.at}
}


\maketitle

\keywords{QBF, DQBF, autarkies}



\setcounter{section}{-1}
\section{TODOS}
\label{sec:todos}

\begin{enumerate}
\item For the abstract, can we say more about ``which (D)QCNF benchmarks have which autarkies''?
\item For the abstract, can see say more about ``while even the above very basic cases can help to solve instances''?
  \begin{enumerate}
  \item Are there even instances which are satisfiable under one of the three systems, while they are hard to solver otherwise?
  \end{enumerate}
 \item Selection of a naming convention for the family of benchmarks. The QBFLIB organised them by the submitted author's name. For example one such directory is: Kronegger-Pfandler-Pichler. It has two subdirectories, bomb and dungeon. What should we use? The paper \href{https://www.react.uni-saarland.de/publications/cav2017-caqe.pdf}{On Expansion and Resolution in CEGAR Based QBF Solving} use bomb and dungeon as the name of the family in the table.
 \item The above paper do not consider all the instances but only those that are solved by any solver. Find which instances are solver by a solver from the corresponding benchmark! Is it possible? Our strategy can be consider all instances (not just restricted to solved ones) and see if reduced instance that was not previously solved can be solved by the solver (is there cases the other way around).
 \item Oliver can you suggest the structure of the experiments table that you have in mind. I had in mind was Table3. What should be the comparison parameter? Average time solving the instances? no. of instance solved out of total?
\end{enumerate}



\section{Introduction}
\label{sec:intro}



\section{Experiments}
\label{sec:experiments}

XXX.
We left out a few families from the QBFLIB to meet our computational constraints. XXX

\begin{table}
\caption{Families of (D)QCNF instances and the corresponding autarky count}\label{tab:aut-count}
\begin{tabular}{c|l|c|c|c|c}

\hline
   \multirow{2}{1cm}{Sr.No} &  \multirow{2}{3cm}{Family} &  \multirow{2}{2cm}{Total instances} & \multicolumn{3}{c}{Autarky system} \\
    % \hline
    % \textbf{Inactive Modes} & \textbf{Description}\\
    \cline{4-6}
  & & & E1-Autarky & UA1-Autarky & E1+A1-Autarky \\
    %\hhline{~--}
    
\hline

1. & DQBF-track'18 &  334 & 0 & 4 [2] & 4 \\ \hline
2. & PCNF-track'18 &  463 & 186 &  &  \\ \hline

% Doable Small+ Low Autarky
3. & Akshay-Chakraborty-John-Shah-Rabe &  44 & 42 &  &  \\ \hline

% No E1, small+medium in size
4. & Amendola-Ricca-Truszczynski & 1933 & 0 &  &  \\ \hline

%% No E1 Check, medium in size
5. & Ansotegui &  38 & 0 &  &  \\ \hline

% Medium (Very big and few small) Low Autarky, choti choti 
6. & Ayari &  71 & 21  &  &  \\ \hline

% Small +, Low Autarky
7. & Basler &  713 & 711 & 0 & 711 \\ \hline

%% No E1 Check, Small + to medium!
8. & Biere &  738 & 0 &  &  \\ \hline

% YUGEEE but some Autarky
9. & Cashmore-Fox-Giunchiglia &  150 & 12 &  &  \\ \hline 

%% No History Check: Very SMALL TARGET!!, Complete E 1 and A1 bith
10. & Castellini &  169 & 162 &  &  \\ \hline

%% No E1 Check Small +m Complete the E1 redution run
11. & Chen &  200 & 2 &  &  \\ \hline

% No E1 Check Small _ Medium 
12. & Diptarama-Jordan-Shinohara &  180 & 0 &  &  \\ \hline

% No history Check Small TARGET, Run E1
13. & Egly-Seidl-Tompits-Woltran-Zolda &  1080 &  &  &  \\ \hline

% No History Check Medium, Run E1
14. & Faber-Leone-Maratea-Ricca &  1060 &  &  &  \\ \hline 

% Big but have some autarkies
15. & Gent-Rowley &  672 & 672 & 0 & 672 \\ \hline

% Medium Big Small Combination but have some autarkies
16. & Herbstritt &  478 & 400 &  &  \\ \hline

17. & Jordan-Kaiser &  4608 & 566 [326] &  &  \\\hline

%% No  E1 Check Small +
18. & Katz &  20 & 0 &  &  \\ \hline

% No E1 Check Small 
19. & Klieber & 50 & 0 &  &  \\ \hline

% Small Good Autarkies E1 not A1 TARGET!	
20. & Kontchakov&  136 & 135 &  &  \\ \hline

% BIg Excelelnt Autarkies E1 
21. & Kronegger-Pfandler-Pichler &  1700 & 1700  & 0  & 1700 \\ \hline 

% No E1 Check Small 
22. & Lahiri-Seshia & 3 & 0 &  &  \\ \hline

% No E1 Check  Medium
23. & Lee-Jiang &  5 &  0 &  &  \\ \hline

% SAT :) Check!!
24. & Letz & 14 & 5 [5]  &  &  \\\hline

% Small :) No E1
25. & Ling &  8 & 0 &  &  \\ \hline

% Yuge low Autarky 
26. & Mangassarian-Veneris &  171 & 17 &  &  \\ \hline

%% Small+ low Autarky Target
27. & MayerEichberger-Saffidine &  312 & 312 & 0 & 312 \\\hline

%%Small to big, No E1!
28. & Messinger &  63 & 0 &  &  \\ \hline

%%Small to medium, Low autakr
29. & Miller-Marin & 2632 & 2613 &  &  \\ \hline

30. & Miller-Scholl & 198 & 198 &  &  \\ \hline

%%Small to big, Low Autarky, Check again E1
31. & Mneimneh-Sakallah &  202 & 6 &  &  \\ \hline

% Medium but too many Can Skip this Family!! No E1!
32. & Narizzano &  4000 & &  &  \\ \hline 

% small a few rest BIG , No E1!
33. & Palacios & 24 & 0 &  &  \\ \hline

% Small target!! No E1!  Medium
34. & Pan &  384 & 378 &  &  \\ \hline

% No E1, Small check A1
35. & Peitl & 10 & 0 &  &  \\\hline

% Small to medium, Low Autarkies
36. & qbfeval12 & 17 & 0 &  &  \\ \hline

% Mediumbut have low autarkies
37. & Rabe &  17 &  &   \\ \hline

% Small + size, Small Autarkies
38. & Rintanen &  131 & 46 [2] &  &  \\\hline

%% Medium to Big (a few small), Low Autarky 
39. & Sauer-Reimer &  928 & 928 & 0 & 928 \\ \hline

% Check Small, low autarky, Recalculate some changes/
40. & Scholl-Becker & 64 &  &  &  \\ \hline

% medium, Low Autarkies  	
41. & Seidl &  200 &  &  &  \\ \hline

% Some SAT Some RED!! Small, recheck E1
42. & Tacchella &  336 & 53 [23] &  &  \\ \hline 

% Big Medium, No E1
43. & Tentrup & 1354 & 0  &  &  \\ \hline \hline

Total &  & 26010 &  &  &  \\  \hline \hline
 

\hline
\end{tabular}
\end{table}


We randomly sample from the set of autarky-reduced instances. To represent each family of benchmarks evenly, we normalise the selection. We take the relative contribution of each family to the total number of 2,000 autarky-reduced instances. XXX

\begin{table}
\caption{Comparison of run time of reduced instance with the original instance}\label{tab:solver-cmp}
\begin{tabular}{l|l|l|l|l}

\hline
    \multirow{2}{3cm}{Family}  & \multicolumn{4}{c}{(D)QCNF Solver} \\
    % \hline
    % \textbf{Inactive Modes} & \textbf{Description}\\
    \cline{2-5}
   & (d)-Caqe & DepQbf & Cute & HQSpre  \\
    %\hhline{~--}
    
\hline

DQBF-track'18 (4) &  4 & 0 & 2 & 4 \\

\hline
\end{tabular}
\end{table}

\begin{table}
\caption{Families of (D)QCNF instances and the corresponding autarky count}\label{tab:aut-results}
\begin{tabular}{l|l|l|l|l|l|l|l|l}

\hline
    \multirow{2}{3cm}{Family} &  \multirow{2}{2cm}{Total instances} & \multicolumn{3}{c}{Autarky system} & \multicolumn{4}{c}{(D)QCNF Solver} \\
    % \hline
    % \textbf{Inactive Modes} & \textbf{Description}\\
    \cline{3-9}
   & & E1 & A1 & E1+A1 & Caqe& Depqbf & Cute & HQSpre\\
    %\hhline{~--}
    
\hline

DQBF-track'18 &  334 & 0 & 2 & 4 \\ \hline
PCNF-track'18 &  463 & 168 & 82 & 174 \\ \hline
Baseler & 71 & 68 & 40 & 71 \\
 

\hline
\end{tabular}
\end{table}

\section{Conclusion}
\label{sec:conclusion}







\end{document}

%%% Local Variables:
%%% mode: latex
%%% TeX-master: t
%%% End:
