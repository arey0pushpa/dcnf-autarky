% Oliver Kullmann, 22.2.2020 (Swansea)
% The bibtex-entry of this paper is \cite{KullmannShukla2020}.
% Submitted to SAT 2020:
% XXX

% Switching between report- and journal-version: uncomment the appropriate
% block of definitions of \Schrift, \Liste.

\documentclass[runningheads]{llncs}

\usepackage{enumerate}
\usepackage[active]{srcltx}
\usepackage[all]{xy}

\usepackage{xr}
\usepackage{amsmath, amssymb}
\usepackage[normalem]{ulem}
\usepackage{lmodern}
\usepackage{breakcites}

\usepackage{multirow}
\usepackage{hyperref}

\usepackage{todonotes}
\newtheorem{prop}{Proposition}

\DeclareMathOperator{\Aaut}{A}
\DeclareMathOperator{\Eaut}{E}

\DeclareMathOperator{\idg}{idg}

\DeclareMathOperator{\varess}{\var_{es}} % essential variables
\DeclareMathOperator{\nessv}{\mathit{n}_{es}} % number of essential variables

\DeclareMathOperator{\amo}{AMO}
\newcommand{\autks}{\textsc{Aut-Search}}
\newcommand{\solvereval}{\textsc{Solver-Eval}}
\newcommand{\caqe}{\textsc{CAQE}}
\newcommand{\depqbf}{\textsc{DepQBF}}
\newcommand{\qute}{\textsc{Qute}}
\newcommand{\hqspre}{\textsc{HQSpre}}
\newcommand{\autfinder}{\textsc{Autark-Finder}}


\begin{document}

\title{Autarkies for DQBF}

\author{Oliver Kullmann\inst{1} \and Ankit Shukla\inst{2}}
\authorrunning{O.~Kullmann, A.~Shukla}

\institute{Swansea University, Swansea, UK\\
  \email{O.Kullmann@Swansea.ac.uk}
\and
    Johannes Kepler University, Austria\\
    \email{ankit.shukla@jku.at}
}

\maketitle

\begin{abstract}
We survey the solving approaches for DQBF, preprocessing and inprocessing techniques, and further advances in DQBF solving.
\end{abstract}

\keywords{QBF, DQBF, autarkies}

\setcounter{section}{-1}
\section{TODOS}
\label{sec:todos}

\begin{enumerate}
\item Compose the literature on DQBF applications.
\end{enumerate}


\section{Introduction}
\label{sec:intro}

In logic a branching quantifier, also called a \textbf{Henkin quantifier}, finite partially ordered quantifier or nonlinear quantifier, is a finite partially ordering 
\[ \langle Qx_{1}\dots Qx_{n}\rangle \]
of quantifiers for $Q  \in \{\forall,\exists\}$.
	
The simplest Henkin quantifier, $Q_{H}$ is :
\begin{displaymath}
\binom{\forall x_{1} \exists y_{1}}{\forall x_{2} \exists y_{2}} \, \phi(x_{1},x_{2},y_{1},y_{2})
\end{displaymath}

Branching quantification first appeared in a 1959 conference paper of Leon Henkin~\cite{henkin1961some}.
Systems of partially ordered quantification are intermediate in strength between first-order logic and second-order logic.
	
In computer science literature the term \textbf{DQBF} was first used in \cite{peterson1979multiple} to model multiple person (team) games of incomplete information.
The generalization of the alternation machines (nondeterministic Turing Machines with existential and universal quantifiers alternation, conceptually similar to QBF) to classes of machines namely multiple person alternation machines (conceptually DQBF).

\section{Multiple player games of incomplete information \cite{petersonr79, peterson2001lower}}

The term DQBF and the proof of DQBF being NEXPTIME-complete were first presented in \cite{peterson1979multiple} and then later in \cite{peterson2001lower}.

Both the works generalize alternation machines \cite{chandra1981lj} and \cite{fraenkel78} (extensions of non-deterministic machines with existential and universal choices) to model multiple player games of incomplete information.
Incomplete information means the players do not have common knowledge of the game being played.
The common knowledge can be payoffs, who the other players are, what moves are possible, what opponent knows, and what he knows I know, etc.

A multiple person (team) game consists of $k+1$ players, $\{0, 1,2, ... ,k\}$ divided into two teams, Team A and Team B.
Team A is our team of preference.
One property of these games is that the size of Team A will be the determinate of the complexity of the game and therefore Team B need only be one player.
So we identify Team A with Players $1,...,k$ and Team B with Player 0.

The use of the word succinct means no more than squaring in size. Consider a DQBF
\[
\forall X_1,X_2 \, \exists Y_1(X_1) \, \exists Y_2(X_2) \,\, F (X_1, X_2, Y_1, Y_2)
\]
this formula do not have a succinct QBF representation.

The formula above corresponds to a three-player game (a modified version is presented here, using \cite{hearn06}) with Players \{0,1,2\}.
There are two teams, A and B.
The input formula $F$ is a CNF over variables $X_1 \cup Y_1 \cup X_2 \cup Y_2$.
The objective is to decide the A Player has a winning strategy.
The Player 0 chooses an assignment of the variables in $X_1 \cup X_2$.
Player 1 then chooses an assignment for $Y_1$, then Player 2 chooses an assignment for $Y_2$.
Player $i$ only sees the assignments of $X_i$ and $Y_i$. Team A (B) wins if the $F$ is true (false).

Both the work \cite{petersonr79, peterson2001lower} is theoretical in nature. The results applied to relative succinctness and power questions of finite state machines and complexity questions of parallel finite state machines.
\cite{peterson2001lower} provide matching lower bounds for the decision algorithms for these games.

\section{First solving approach: DQDPLL}
The first solving approach was presented in~\cite{frohlich2012dpll}.
%
This was adaptation of QDPLL from QBF to DQBF, e.g., unit propagation, clause
learning, universal reduction, watched literals, etc.
%

The motivation was to solve the quantifier-free bit-vector formulas (QF\_BV) by translating it to DQBF (both belongs to same complexity class).

\section{First Application: PEC Problems}
The application of partial equivalence checking (PEC) of circuits transformed as a DQBF satisfiability was presented in~\cite{gitina2013equivalence}.
%
The presented algorithm solve DQBF based on variable elimination (\cite{biere2004resolve}).

\section{First publicly available solver}
The technique presented in~\cite{finkbeiner2014fast} is similar to BMC encoding of QBF and can only solve UNSAT problems.  

\section{Complete solver: iDQ}
The iDQ solver presented in~\cite{frohlich2014idq} adapts and extends the Inst-Gen approach (solving approach for EPR logic, fragment of first order logic) to DQBF.

\section{A new solver: HQS}
The work of~\cite{gitina2015solving} present an improved expansion-based solver, HQS.
%
The solver expands DQBF to QBF, eliminates the minimum set of variables that cause non-linear
dependencies and uses AIGs to detect units and pure literals.

\section{Preprocessing and inprocessing}
The work of~\cite{wimmer2015preprocessing} generalize preprocessing techniques for QBF to DQBF. 
%
The technique includes, blocked clause elimination, equivalence reasoning, structure extraction, and variable elimination by resolution.

In further work~\cite{wimmer2017hqspre} presents a preprocessor HQSpre for both QBF's and DQBF's.
%
The techniques, variable elimination routines, clause elimination routines, clause strengthening routines, dependency schemes and gate rewriting.

\section{Dependency}
\todo{Only read overview of the papers in this section: read completely}The work by~\cite{wimmer2016dependency} analyse the dependency
schemes proposed for QBF to apply on DQBF. 
%
The paper gives correctness proof of the few that are sound for DQBF. 
%

The \cite{wimmer2017dqbf} eliminate of dependencies of DQBF to convert it to an equisatisfiable QBF.
%
The technique is implemented in HQS.

% which result from the application of dependency schemes to the formula and can be added to or removed from the formula at no cost


The work ~\cite{beyersdorff2018reinterpreting} proposed an approach which improve the relationship between DQBF and QBF
dependency schemes.
%
Core of the paper is the method that gives rise to soundness results across different calculi.

\section{Proof systems}
The work~\cite{beyersdorff2016lifting} examines the lifting existing resolution systems of QBF to DQBF.
%
QBF have a strict chain of proof systems, where Q-Res $<$ IR-calc $<$ IRM-calc.
%

In the case of DQBF the obvious adaptations of Q-Res and universal resolution turns out to be too weak: they are not complete. 
%
The obvious adaptation of IR-calc has
the right strength: it is sound and complete. 
%
The IRM-calc and long-distance resolution is too strong: it is not sound any more.

\todo{Incomplete and only intro}The work of~\cite{rabe2017resolution} presents a sound and complete proof system for DQBF.
%
The system has three rules, resolution, universal reduction, and a new proof rule called \textit{fork extensions}.

The definition of resolution admits universal variables as pivots...

\section{Certification perspective and Skolem functions}

%\begin{enumerate}
The work of Henkin Quantifiers and Boolean Formulae~\cite{balabanov2012henkin} investigates DQBF's properties like negation, complement, formula expansion, and prenex, non-prenex form conversion. 
%
The paper used two forms to describe prenex DQBF, Skolem (S-Form) and Herbrand (H-Form) form:

\begin{align}
\Phi_{S} \,\, = \,\, \forall x_{1}... \forall x_{n} \exists y_{1_{(S_{1})}}...\exists y_{m_{(S_{m})}} . \phi \label{eq1}\\
\Phi_{H}  \,\, = \,\, \forall x_{1_{(H_{1})}}... \forall x_{n_{(H_{n})}} \exists y_{1}...\exists y_{m} . \phi \label{eq2}
\end{align}

Additional subscripts on the quantified variables is just another way to express dependencies, $S_{i} \subseteq A$ and $H_{i} \subseteq E$.

In the case of S-Form, $\exists y_{1_{(S_{1})}}$ means  existential variable $y_{1}$ depend on the universal variables set $S_{1}$. 
%
Similarly, for the H-form, universal quantifiers in the prefix are annotated with the existential variables they depend on. 

%Semantically, the truth and falsity of a DQBF is interpreted as existence of Skolem (in the case of \ref{eq1}) and Herbrand fucntion (in the case of \ref{eq2}). 
Skolem functions serve as the model to a true S-form DQBF whereas Herbrand functions serve as the countermodel to a false H-form DQBF.

Most important property discussed is the notion of \textbf{negation} and \textbf{complement}.
%
In the case of negation ($\neg$) the quantifiers are flipped and propositional formula is negated but no change in the dependency set and it's position.

\begin{align}
\neg \Phi_{S} \,\, =   \,\, \exists x_{1}... \exists x_{n} \forall y_{1_{(S_{1})}}...\forall y_{m_{(S_{m})}} . \neg \phi \label{eq3}
\end{align}

In the case of complement ($\sim$) the position is shifted from one quantifier type to another in the prefix. Here, $H^{\prime}_{i} = \{ y_{i} \in E \, | \, x_{i} \notin S_{j} \}$

\begin{align}
\sim \Phi_{S} \,\, =   \,\, \exists x_{1_{(H^{\prime}_1)}}... \exists x_{1_{(H^{\prime}_n)}} \forall y_{1}...\forall y_{m} . \neg \phi \label{eq4}
\end{align}

Interestingly the following proposition holds, 
\begin{prop}
	DQBF under the negation operation obeys the law of excluded middle but under the complement operation it does not.
\end{prop} 

In the journal paper ``A certification perspective of DQBF"~\cite{balabanov2014henkin} additionally a generalized resolution rule for DQBF evaluation is presented. Although,  it is shown to be sound, but incomplete.

Two central references in both are~\cite{bubeck2006dependency} and ~\cite{bubeck2010model} in introduction section referring that S-form DQBF can be converted to an equivalent QBF by formula expansion on the universal variables.

%\end{enumerate}

The work by~\cite{wimmer2016skolem} obtains Skolem functions from an elimination-based DQBF solver HQS, while taking preprocessing steps into account.
%
The size of the Skolem functions is optimized by don’t-care minimization using Craig interpolants and rewriting techniques.


\section{Experiments}\label{sec:experiments}
We perform two sets of experiments. Finding autarkies on the selected benchmarks and evaluating the effect of the autarky-reduction on a particular solver.

We performed autarky search experiments on benchmarks from both the DQBF- (334 instances) and  PCNF-track (463 instances) of QBFEVAL'18 \cite{Qbfeval18}.
%
From the QBFLIB benchmark \cite{Qbflib}, we removed non-PCNF instances (instances in qcir format) and selected instances in PCNF format (25,522 instances) for the experiment.

% We also performed autarky search experiments on  PCNF format (25,650 instances) instances of .
%QBFEVAL'18 benchmarks (total 1234 instances) and .

%From the QBFEVAL'18 benchmarks \cite{Qbfeval18}, we selected instances in PCNF format and removed non-PCNF instances (instances from QCIR-track). In total, we considered 334 instances from DQBF-track and 463 instances from PCNF-track.
%Similarly, from QBFLIB benchmarks \cite{Qbflib} we removed all instances that are not in PCNF. In total, we considered 25,650 instances from QBFLIB to perform the autarky search.

We selected four (D)QBF solvers to perform the evaluation of autarky-reduction. Top three solvers of QBFEVAL'19 PCNF-track competition; \caqe \cite{DBLP:conf/fmcad/RabeT15}, \depqbf \cite{DBLP:conf/cade/LonsingE17}, \qute \cite{DBLP:journals/jsat/PeitlSS19}\ and the winner of DQBF-track competition, \hqspre \cite{DBLP:conf/tacas/WimmerRM017}.

All the experiments were performed on our cluster where each compute node has
two Intel Xeon E5-2620 v4 CPUs running at 2.10 GHz with turbo-mode disabled.
Time limit was set to 2700 seconds and memory limit to 7 GB.

\subsection{Implementation}
We have implemented the translation for $\Aaut_1$- and $\Eaut_1$-autarky search  (\ref{sec:a1-translation}, \ref{sec:e1-translation}) in a small codebase \autfinder\footnote{https://github.com/arey0pushpa/dcnf-autarky}. The \autfinder\ includes 2,000 lines of C++ code.

The translation requires for each $v \in E$ expressing AMO constraints in equation~\ref{eq:transAMO}, we implemented three encodings, binomial, logarithmic and linear. One of these options can be selected with the $\Aaut_1$ or $\Eaut_1+\Aaut_1$ autarky type. The resulted CNF formula is solved by the lingeling \cite{Biere-SAT-Competition-2017-solvers} SAT solver.


\begin{table}
\caption{Families of (D)QCNF instances and the corresponding autarky count}\label{tab:aut-count}
\begin{tabular}{c|l|c|c|c|c}

\hline
   \multirow{2}{1cm}{Sr.No} &  \multirow{2}{3cm}{Family} &  \multirow{2}{2cm}{Total instances} & \multicolumn{3}{c}{Autarky system} \\
    % \hline
    % \textbf{Inactive Modes} & \textbf{Description}\\
    \cline{4-6}
  & & & E1-Autarky & A1-Autarky & E1+A1-Autarky \\
    %\hhline{~--}
    
\hline

1. & DQBF-track'19 &  334 & 20 & 22 & 22[1] \\ \hline
%2. & PCNF-track'19 &  463 & 186 & 48[3] & 194[4] \\ \hline
2. & PCNF-track'19 &  559 & 200 & 40[4] & 50[6] \\ \hline
% Doable Small+ Low Autarky
3. & Akshay-Chakraborty-J-S-R &  44 & 42 & 4[3] & 44[3] \\ \hline
% No E1, small+medium in size
4. & Amendola-Ricca-Truszczynski & 1933 & &  &  \\ \hline

%% No E1 Check, medium in size
5. & Ansotegui &  38 & &  &  \\ \hline

% Medium (Very big and few small) Low Autarky, choti choti 
6. & Ayari &  71 &  &  &  \\ \hline

% Small +, Low Autarky
7. & Basler &  713 & & &  \\ \hline

%% No E1 Check, Small + to medium!
8. & Biere &  738 & &  &  \\ \hline

% YUGEEE but some Autarky
9. & Cashmore-Fox-Giunchiglia &  150 &  &  &  \\ \hline 

%% No History Check: Very SMALL TARGET!!, Complete E 1 and A1 bith
10. & Castellini &  169 & &  &  \\ \hline

%% No E1 Check Small +m Complete the E1 redution run
11. & Chen &  200 &  &  &  \\ \hline

% No E1 Check Small _ Medium 
12. & Diptarama-Jordan-Shinohara &  180 &  &  &  \\ \hline

% No history Check Small TARGET, Run E1
13. & Egly-Seidl-Tompits-Woltran-Zolda &  1080 &  &  &  \\ \hline

% No History Check Medium, Run E1
14. & Faber-Leone-Maratea-Ricca &  1060 &  &  &  \\ \hline 

% Big but have some autarkies
15. & Gent-Rowley &  672 &  &  &  \\ \hline

% Medium Big Small Combination but have some autarkies
16. & Herbstritt &  478 & &  &  \\ \hline

17. & Jordan-Kaiser &  4608 & &  &  \\\hline

%% No  E1 Check Small +
18. & Katz &  20 &  &  &  \\ \hline

% No E1 Check Small 
19. & Klieber & 50 &  &  &  \\ \hline

% Small Good Autarkies E1 not A1 TARGET!	
20. & Kontchakov&  136 & &  &  \\ \hline

% BIg Excelelnt Autarkies E1 
21. & Kronegger-Pfandler-Pichler &  1700 &  &   & \\ \hline 

% No E1 Check Small 
22. & Lahiri-Seshia & 3 & &  &  \\ \hline

% No E1 Check  Medium
23. & Lee-Jiang &  5 &  &  &  \\ \hline

% SAT :) Check!!
24. & Letz & 14 &  &  &  \\\hline

% Small :) No E1
25. & Ling &  8 & &  &  \\ \hline

% Yuge low Autarky 
26. & Mangassarian-Veneris &  171 & &  &  \\ \hline

%% Small+ low Autarky Target
27. & MayerEichberger-Saffidine &  312 & & & \\\hline

%%Small to big, No E1!
28. & Messinger &  63 & &  &  \\ \hline

%%Small to medium, Low autakr
29. & Miller-Marin & 2632 & &  &  \\ \hline

30. & Miller-Scholl & 198 & &  &  \\ \hline

%%Small to big, Low Autarky, Check again E1
31. & Mneimneh-Sakallah &  202 & &  &  \\ \hline

% Medium but too many Can Skip this Family!! No E1!
32. & Narizzano &  4000 & &  &  \\ \hline 

% small a few rest BIG , No E1!
33. & Palacios & 24 &  &  &  \\ \hline

% Small target!! No E1!  Medium
34. & Pan &  384 & &  &  \\ \hline

% No E1, Small check A1
35. & Peitl & 10 &  &  &  \\\hline

% Small to medium, Low Autarkies
36. & qbfeval12 & 17 & &  &  \\ \hline

% Mediumbut have low autarkies
37. & Rabe &  17 &  & &  \\ \hline

% Small + size, Small Autarkies
38. & Rintanen &  131 & &  &  \\\hline

%% Medium to Big (a few small), Low Autarky 
39. & Sauer-Reimer &  928 & &  & \\ \hline

% Check Small, low autarky, Recalculate some changes/
40. & Scholl-Becker & 64 &  &  &  \\ \hline

% medium, Low Autarkies  	
41. & Seidl &  200 &  &  &  \\ \hline

% Some SAT Some RED!! Small, recheck E1
42. & Tacchella &  336 & &  &  \\ \hline 

% Big Medium, No E1
43. & Tentrup & 1354 &  &  &  \\ \hline \hline

Total &  & 26010 &  &  &  \\  \hline \hline
 

\hline
\end{tabular}
\end{table}


\subsection{Results}
We use the name of the author to refer to the submitted families of the benchmarks. This is the structure the QBFLIB is organised.

In Table~\ref{tab:aut-count} we present the complete results of the autarky search. For example, out of 334 instances in the DQBF track of QBFEVAL'18, 330 instances are $\Eaut_1+\Aaut_1$-lean (have no non-trivial $\Eaut_1$- or $\Aaut_1$-autarky). From the remaining 4 instances, all are $\Eaut_1$-lean, one is $\Aaut_1$ satisfiable, one is $\Eaut_1+\Aaut_1$ satisfiable, and the remaining two instances are not $\Eaut_1+\Aaut_1$ satisfiable, but allow $\Aaut_1$-autarkies elimination.
%
There are a few families of the benchmarks like \textit{Castellini} that has a healthy number of autarkies of all types, $\Eaut_1,\Aaut_1,\Eaut_1+\Aaut_1$. On the other hand there are families like \textit{Tentrup} and  with no autarkies at all. There are total 714 (autarky satisfiable) instances which can be directly solved as a part of autarky cleanup.
% in both cases around 70\% of variables and clauses (while adding E1 does not change this).



Table~\ref{tab:sat-solver} shows the number of $\Eaut_1+\Aaut_1$-autark-SAT instance solved by the corresponding (D)QCNF solvers. There are a few instances that are autarky-satisfiable but the solver was unable to solve them in the given time. In particular there was a large set of trivially satisfiable autarky instances which \hqspre\ was unable to solve. This is relevant to our case as our focus is primarily DQCNF and provide a hint that $\Eaut_1+\Aaut_1$ autarky search is not already covered by other pre-processing techniques.

Next out of the total 10,708 autarky instances we select at random around 1100 instances to register the effect of the autarky reduction on the solvers. The selection
%
%Out of total 463 instances of the PCNF-track of QBFEval'18 we found that 269 instances are $\Eaut_1+\Aaut_1$ Lean.
Table~\ref{tab:solver-eval} present the result of solver running on the original instance vs on the autarky-reduced instance. We found XXX

%We randomly sample from the set of autarky-reduced instances. To represent each family of benchmarks evenly, we normalise the selection. We take the relative contribution of each family to the total number of 2,000 autarky-reduced instances. XXX


\begin{table}
	\caption{Total $\Eaut_1+\Aaut_1$ Autark-SAT instance solved by the corresponding (D)QCNF solvers}\label{tab:sat-solver}
	\begin{tabular}{l|c|c|c|c}
		
		\hline
		\multirow{2}{3cm}{Family}  & \multicolumn{4}{c}{(D)QCNF Solver} \\
		% \hline
		% \textbf{Inactive Modes} & \textbf{Description}\\
		\cline{2-5}
		& Caqe & DepQbf & Qute & HQSpre  \\
		%\hhline{~--}
		
		\hline
		
		SAT-instances (500) &  474 & 484 & 485 & 392 \\
		
		\hline
	\end{tabular}
\end{table}

\begin{table}
	\caption{Comparison of the (D)QCNF instances solved without and with autarky reduction. \emph{Orig} represent the original instance and \emph{Red} represent the autarky reduced formula. Number inside () represents uniquely solved instances.  }\label{tab:solver-eval}
	\begin{tabular}{l| c |c c c }
		
		\hline
		\multirow{2}{2.5cm}{Solvers} &  \multirow{2}{1cm}{Type} & \multicolumn{3}{c}{Instances} \\
		\cline{3-5}
		& & Random-255 [$\Aaut_1$] & Random-336 [$\Eaut_1+\Aaut_1$] & Random-500 [$\Eaut_1$] \\\hline
		% \hline
		% \textbf{Inactive Modes} & \textbf{Description}\\
		
		%   & & \multicolumn{2}{c}{Caqe} &\multicolumn{2}{c}{Depqbf} & \multicolumn{2}{c}{Qute}& \multicolumn{2}{c}{HQSpre}\\  \cline{3-10}
		%     & & Orig & Red & Orig & Red & Orig & Red & Orig & Red \\
		%\hhline{~--}
		
		\hline
		\multirow{2}{*}{\caqe} & Orig & 193(4) & 228(2) & 381(2) \\
		& Red & 196(7) & 230(4) & 380(1)\\\hline
		\multirow{2}{*}{\depqbf} & Orig & 184(2) & 231(1) & 365(7)\\
		& Red & 183(1)& 232(2)& 367(9)\\\hline
		\multirow{2}{*}{\qute} & Orig & 117(1) & 205(0) & 283(3)  \\
		& Red  & 124(8) & 210(5) & 281(1) \\\hline
		\multirow{2}{*}{\hqspre} & Orig & 96(0) &251(0) & 279(4) \\
		& Red  & 98(2)& 254(4) & 280(5) \\
		% &
		%Qute &
		%HQSpe &
		%Random set-1 &  326 & 228(2) & 230(4) & 231(1) & 232(2) & 205(0) & 210(5) & 251(0)& 254(4) \\ \hline
		%%PCNF-track'18 &  463 & 168 & 82 & 174 \\ \hline
		%Random set-2 ($\Eaut_1+\Aaut_1$)&  255 & 193(4) & 196(7) & 184(2)& 183(1) & 117(1) & 124(8) & 96(0)& 98(2) \\\hline
		%Random set-3 (Only E1) &  500 & 381(2) & 380(1) & 365(7) & 367(9) & 283(3) & 281(1) & 279(4) & 280(5) \\
		
		\hline
	\end{tabular}
\end{table}

\section{Conclusion}
\label{sec:conclusion}

\bibliographystyle{apalike}
\bibliography{DQBF_Autarky}

\end{document}

%%% Local Variables:
%%% mode: latex
%%% TeX-master: t
%%% End:
